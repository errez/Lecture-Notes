\section{강의노트 템플릿} 

본 문서는 \LaTeX 강의노트에 대한 것이며, pdflatex를 이용하여 윈도우 10에서 빌드되었다. 

\subsection{본 템플릿에 대한 간단한 소개}

\paragraph{문서 구조} 본 템플릿을 사용할 시, 크게 신경쓸 부분은 없을 것이다. 기본적으로는 html과 비슷한 구조이나, 환경 시작은 {\textbackslash}begin\{...\}로, 끝은 {\textbackslash}end\{...\}로 끝낸다. section이나 subsection, subsubsection은 굳이 열거나 닫지 않아도 된다. 명령어는 \textbackslash를 앞에 붙여서 나타낸다. 본 문서의 TeX 코드를 보면 대략적으로 어떤 식으로 쓰는지에 대해서 알 수 있으리라 생각된다. 각 강의를 들은 후 Summary.tex에 \textbackslash include \{파일이름\}으로 추가한 후 Summar.tex를 pdflatex로 두 번 컴파일하면 된다. 각주는 이렇게\footnote{footnote 명령어} 달 수 있다. 일반적인 \LaTeX 사용법에 대해서는 \href{http://legacy-wiki.dgoon.net/doku.php?id=latex:latex}{LaTeX wiki}나 \href{http://ftp.isu.edu.tw/pub/Unix/CTAN/info/lshort/korean/lshort-kr.pdf}{LaTeX 메뉴얼}을 참고하는 것을 추천하다.

\paragraph{소스 코드} 본 템플릿에서는 파이썬, 자바, C 소스 코드를 지원\footnote{다른 언어 지원이 필요하면 \href{mailto:principia\_12@kaist.ac.kr}{관리자} 에게 문의}한다. 
\begin{itemize} 

\item{Python} 

\begin{Python} 
def main:
    print('Hello, World!')
    
    return 0
    
\end{Python} 

\item{C}


\begin{C} 
#include <stdio.h>

int main(int argc, const char * argv[]) 
{
    printf("Hello, World!\n");
    return 0;
}    
\end{C} 

\item{Java}

\begin{Java} 
public class HelloWorld {
    public static void main(String[] args) {
        System.out.println("Hello, world!");
    }
}
\end{Java} 

\end{itemize}

    